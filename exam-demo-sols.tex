% arara: xelatex
\documentclass[12pt]{article}

% \usepackage{physics}


\usepackage{tikzducks}

\usepackage{tikz} % картинки в tikz
\usepackage{microtype} % свешивание пунктуации

\usepackage{array} % для столбцов фиксированной ширины

\usepackage{indentfirst} % отступ в первом параграфе

\usepackage{sectsty} % для центрирования названий частей
\allsectionsfont{\centering}

\usepackage{amsmath, amsfonts, amssymb} % куча стандартных математических плюшек

\usepackage{comment}

\usepackage[top=2cm, left=1.2cm, right=1.2cm, bottom=2cm]{geometry} % размер текста на странице

\usepackage{lastpage} % чтобы узнать номер последней страницы

\usepackage{enumitem} % дополнительные плюшки для списков
%  например \begin{enumerate}[resume] позволяет продолжить нумерацию в новом списке
\usepackage{caption}

\usepackage{url} % to use \url{link to web}


\newcommand{\smallduck}{\begin{tikzpicture}[scale=0.3]
    \duck[
        cape=black,
        hat=black,
        mask=black
    ]
    \end{tikzpicture}}

\usepackage{fancyhdr} % весёлые колонтитулы
\pagestyle{fancy}
\lhead{}
\chead{}
\rhead{Решение демо-версии экзамена}
\lfoot{Методы оптимальных решений}
\cfoot{}
\rfoot{}

\renewcommand{\headrulewidth}{0.4pt}
\renewcommand{\footrulewidth}{0.4pt}

\usepackage{tcolorbox} % рамочки!

\usepackage{todonotes} % для вставки в документ заметок о том, что осталось сделать
% \todo{Здесь надо коэффициенты исправить}
% \missingfigure{Здесь будет Последний день Помпеи}
% \listoftodos - печатает все поставленные \todo'шки


% более красивые таблицы
\usepackage{booktabs}
% заповеди из докупентации:
% 1. Не используйте вертикальные линни
% 2. Не используйте двойные линии
% 3. Единицы измерения - в шапку таблицы
% 4. Не сокращайте .1 вместо 0.1
% 5. Повторяющееся значение повторяйте, а не говорите "то же"



\usepackage{fontspec}
\usepackage{libertine}
\usepackage{polyglossia}

\setmainlanguage{russian}
\setotherlanguages{english}

% download "Linux Libertine" fonts:
% http://www.linuxlibertine.org/index.php?id=91&L=1
% \setmainfont{Linux Libertine O} % or Helvetica, Arial, Cambria
% why do we need \newfontfamily:
% http://tex.stackexchange.com/questions/91507/
% \newfontfamily{\cyrillicfonttt}{Linux Libertine O}

\AddEnumerateCounter{\asbuk}{\russian@alph}{щ} % для списков с русскими буквами
\setlist[enumerate, 2]{label=\asbuk*),ref=\asbuk*}

%% эконометрические сокращения
\DeclareMathOperator{\Cov}{\mathbb{C}ov}
\DeclareMathOperator{\Corr}{\mathbb{C}orr}
\DeclareMathOperator{\Var}{\mathbb{V}ar}

\let\P\relax
\DeclareMathOperator{\P}{\mathbb{P}}

\DeclareMathOperator{\E}{\mathbb{E}}
% \DeclareMathOperator{\tr}{trace}
\DeclareMathOperator{\card}{card}
\DeclareMathOperator{\plim}{plim}
\DeclareMathOperator{\pCorr}{\mathrm{p}\mathbb{C}\mathrm{orr}}

\DeclareMathOperator{\Convex}{Convex}

\newcommand \hb{\hat{\beta}}
\newcommand \hs{\hat{\sigma}}
\newcommand \htheta{\hat{\theta}}
\newcommand \s{\sigma}
\newcommand \hy{\hat{y}}
\newcommand \hY{\hat{Y}}
\newcommand \e{\varepsilon}
\newcommand \he{\hat{\e}}
\newcommand \z{z}
\newcommand \hVar{\widehat{\Var}}
\newcommand \hCorr{\widehat{\Corr}}
\newcommand \hCov{\widehat{\Cov}}
\newcommand \cN{\mathcal{N}}
\newcommand \RR{\mathbb{R}}
\newcommand \NN{\mathbb{N}}
\newcommand{\cF}{\mathcal{F}}
\newcommand{\cH}{\mathcal{H}}
\newcommand{\dBin}{\mathrm{Bin}}


% circled text in math mode
% https://tex.stackexchange.com/questions/67127/creating-a-circled-operator-which-expands-into-a-lozenge
\makeatletter
\newcommand*\comp[2][]{%
  \ensuremath{%
    \mathbin{%
      \mathpalette{\comp@aux{#1}}{#2}%
    }%
  }%
}
\newdimen\comp@unit
\newcommand*{\comp@aux}[3]{%
  #2%
  \mskip.5\thinmuskip\nonscript\mskip-.25\thinmuskip
  \begingroup
    \sbox0{%
      $%
        \m@th % \mathsurround=0pt
        #2% \displaystyle, \textstyle, ...
        \mkern9mu %
      $%
    }%
  \edef\x{\endgroup
    \comp@unit=\the\wd0 %
  }\x
  \tikz[baseline=(char.base)]{%
    \node[
      rectangle,
      draw,
      minimum height=2\comp@unit,
      minimum width=2\comp@unit,
      rounded corners=\comp@unit,
      inner sep=.33\comp@unit,
      line width=.05\comp@unit,
      #1%
    ] (char) {%
      $%
        \m@th % \mathsurround=0pt
        #2% \displaystyle, \textstyle, ...
        \rule{0pt}{\comp@unit}%
        #3%
      $%
    };%
  }%
  \mskip.5\thinmuskip\nonscript\mskip-.25\thinmuskip
}
\makeatother



\begin{document}


\begin{enumerate}
    \item 
    Двойственная задача:
    \[
    \begin{cases}
        24y_1 + 20y_2 -4 y_3 \min \\
        3y_1 + y_2 +2 y_3 \geq 4 \\
        y_1 -3y_3 \geq -1 \\
        -2y_1 + y_2 -y_3 = 4 \\
        y_1 + 2y_2 - y_3 \geq 7 \\
        y_1 \geq 0, y_2 \in \RR, y_3 \geq 0 
    \end{cases}
    \]
    
    
    \item 
    \begin{enumerate}
        \item 
    Двойственная задача:
    \[
    \begin{cases}
    w = 9y_1 + 6y_2 \to \min \\
    y_1 + y_2 \geq 4 \\
    5y_1 + y_2 \geq 12 \\
    y_1 + 8y_2 \geq 18 \\
    y_1 \geq 0, y_2 \geq 0 
    \end{cases}
    \]
    Прямые $\ell_1$, $\ell_2$ и $\ell_3$ пересекаются в одной точке. 
    
    Решение двойственной задачи: $y_1 = 2$, $y_2 = 2$, минимум равен $30$.

    \item В двойственной задаче $y_1 > 0$, поэтому $x_1 + 5x_2 + x_3 = 9$. 
    В двойственной задаче $y_2 > 0$, поэтому $x_1 + x_2 + 8x_3 = 6$.
    
    Решение исходной задачи: $x_3 \in [0; 21/39]$, $x_2 = (3 + 7x_3) / 4$, $x_1 = (21 - 39 x_3) / 4$,  максимум равен $30$. 

    Решение исходной задачи можно также записать в виде $\Convex(A, B)$, где $A = (21/4, 3/4, 0)$, $B = (0, 22/13, 21/39)$.

    \item Сравниваем два варианта:    
    \begin{enumerate}
        \item Решение двойственной задачи сохраняется. 
        Изменение прибыли равно $\Delta \pi = - \Delta b_1 \cdot p + \Delta b_1 \cdot y_1 =  2 \cdot 2 - 2 \cdot 2 = 0$.
        \item Решение двойственной задачи сохраняется. 
        Изменение прибыли равно $\Delta \pi = - \Delta b_2 \cdot p + \Delta b_2 \cdot y_2 =  -3 \cdot 1 + 3 \cdot 2 = 3$.
        Данный вариант выгоднее. 
    \end{enumerate}
\end{enumerate}

    \item 
    \item 
    \begin{tiny}
        \begin{tabular}{ccccccccccccccccccc}
        \toprule
        Грузоподъёмность & $0$ & $1$ & $2$ & $3$ & $4$ & $5$ & $6$ & $7$ & $8$ & $9$ & $10$ & $11$ & $12$ & $13$ & $14$ & $15$ & $16$ & $17$ \\
        \midrule
        $A$ & $0$ & $0$ & $0$ & $4$ & $4$ & $4$ & $8$ & $8$ & $8$ & $12$ & $12$ & $12$ & $16$ & $16$ & $16$ & $20$ & $20$ & $20$ \\
        $A, B$ & $0$ & $0$ & $0$ & $4$ & $4$ & $4$ & $8$ & $\comp{9}/8$ & $\comp{9}/8$ & $9/12$ & $13/12$ & $13/12$ & $13/16$ & $17/16$ & $18/16$ & $18/20$ & $21/20$ & $22/20$ \\
        %$A$, $B$, $C$ & $0$ & $0$ & $5$ & $6$ & $10$ & $13/11$ & $13/15$ & $18/16$ & $19/20$ & $23/21$ & $26/25$ & $28/30$ & $31/30$ & $33/31$ & $36/35$ & & & \\
        %$A$, $B$, $C$, $D$ & $0$ & $0$ & $5$ & $6$ & $10$ & $13$ & $16/15$ & $16/18$ & $21/20$ & $22/23$ & $26/26$ & $29/28$ & $32/31$ & $34/33$ & $37/36$ & & & \\
        \bottomrule
      \end{tabular}
      \end{tiny}

    \begin{enumerate}
        \item 
        \item 
    \[
    \begin{cases}
        4x_a + 9x_b + 10 x_c \to \max \\
        3x_a + 7x_b + 8x_c \leq 17 \\
        x_a, x_b, x_c \in \{0, 1, 2, 3, \ldots \}
    \end{cases}
    \]

    \end{enumerate}

    \item 

    \begin{tabular}{cccccccccc}
        \toprule
        вершина & & & & & & & & &  \\
        \midrule
        $A_1$ & $0$ & $0^*$ & & & & & & &  \\
        $A_2$ & $\infty$ & $6$ & $3$ & $3$ & $3^*$ & & & & \\
        $A_3$ & $\infty$ & $2$ & $2$ & $2^*$ &  & & & & \\
        $A_4$ & $\infty$ & $1$ & $1^*$ &  &  & & & & \\
        $A_5$ & $\infty$ & $\infty$ & $\infty$ & $7$ & $6$ & $6^*$ & & & \\
        $A_6$ & $\infty$ & $\infty$ & $8$ & $8$ & $8$ & $7$ & $7^*$ & & \\
        $A_7$ & $\infty$ & $\infty$ & $9$ & $9$ & $9$ & $9$ & $8$ & $8^*$ & \\
        $A_8$ & $\infty$ & $\infty$ & $\infty$ & $\infty$ & $\infty$ & $11$ & $9$ & $9$ & $9^*$ \\
        \bottomrule
    \end{tabular}
    \begin{enumerate}
        \item Оптимальные маршруты:   
        \[
        A_1 \overset{1}{\to} A_4 \overset{2}{\to} A_2 \overset{3}{\to} A_5 
        \overset{1}{\to} A_6 \overset{2}{\to} A_8, \quad        
        A_1 \overset{1}{\to} A_4 \overset{2}{\to} A_2 \overset{3}{\to} A_5 
        \overset{1}{\to} A_6 \overset{1}{\to} A_7 \overset{1}{\to} A_8, 
        \]
        стоимость равна $9$.
        \item $A_2 \overset{3}{\to} A_5 \overset{1}{\to} A_6 \overset{1}{\to} A_7$, стоимость равна $5$.
    \end{enumerate}

\end{enumerate}




\end{document}

