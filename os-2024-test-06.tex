% arara: xelatex
\documentclass[12pt]{article}

% \usepackage{physics}


\usepackage{tikzducks}

\usepackage{tikz} % картинки в tikz
\usepackage{microtype} % свешивание пунктуации

\usepackage{array} % для столбцов фиксированной ширины

\usepackage{indentfirst} % отступ в первом параграфе

\usepackage{sectsty} % для центрирования названий частей
\allsectionsfont{\centering}

\usepackage{amsmath, amsfonts, amssymb} % куча стандартных математических плюшек

\usepackage{comment}

\usepackage[top=2cm, left=1.2cm, right=1.2cm, bottom=2cm]{geometry} % размер текста на странице

\usepackage{lastpage} % чтобы узнать номер последней страницы

\usepackage{enumitem} % дополнительные плюшки для списков
%  например \begin{enumerate}[resume] позволяет продолжить нумерацию в новом списке
\usepackage{caption}

\usepackage{url} % to use \url{link to web}


\newcommand{\smallduck}{\begin{tikzpicture}[scale=0.3]
    \duck[
        cape=black,
        hat=black,
        mask=black
    ]
    \end{tikzpicture}}

\usepackage{fancyhdr} % весёлые колонтитулы
\pagestyle{fancy}
\lhead{Имя:}
\chead{Группа:}
\rhead{Праздник \smallduck$\times 6$}
\lfoot{Методы оптимальных решений}
\cfoot{}
\rfoot{}

\renewcommand{\headrulewidth}{0.4pt}
\renewcommand{\footrulewidth}{0.4pt}

\usepackage{tcolorbox} % рамочки!

\usepackage{todonotes} % для вставки в документ заметок о том, что осталось сделать
% \todo{Здесь надо коэффициенты исправить}
% \missingfigure{Здесь будет Последний день Помпеи}
% \listoftodos - печатает все поставленные \todo'шки


% более красивые таблицы
\usepackage{booktabs}
% заповеди из докупентации:
% 1. Не используйте вертикальные линни
% 2. Не используйте двойные линии
% 3. Единицы измерения - в шапку таблицы
% 4. Не сокращайте .1 вместо 0.1
% 5. Повторяющееся значение повторяйте, а не говорите "то же"



\usepackage{fontspec}
\usepackage{libertine}
\usepackage{polyglossia}

\setmainlanguage{russian}
\setotherlanguages{english}

% download "Linux Libertine" fonts:
% http://www.linuxlibertine.org/index.php?id=91&L=1
% \setmainfont{Linux Libertine O} % or Helvetica, Arial, Cambria
% why do we need \newfontfamily:
% http://tex.stackexchange.com/questions/91507/
% \newfontfamily{\cyrillicfonttt}{Linux Libertine O}

\AddEnumerateCounter{\asbuk}{\russian@alph}{щ} % для списков с русскими буквами
\setlist[enumerate, 2]{label=\asbuk*),ref=\asbuk*}

%% эконометрические сокращения
\DeclareMathOperator{\Cov}{\mathbb{C}ov}
\DeclareMathOperator{\Corr}{\mathbb{C}orr}
\DeclareMathOperator{\Var}{\mathbb{V}ar}

\let\P\relax
\DeclareMathOperator{\P}{\mathbb{P}}

\DeclareMathOperator{\E}{\mathbb{E}}
% \DeclareMathOperator{\tr}{trace}
\DeclareMathOperator{\card}{card}
\DeclareMathOperator{\plim}{plim}
\DeclareMathOperator{\pCorr}{\mathrm{p}\mathbb{C}\mathrm{orr}}


\newcommand \hb{\hat{\beta}}
\newcommand \hs{\hat{\sigma}}
\newcommand \htheta{\hat{\theta}}
\newcommand \s{\sigma}
\newcommand \hy{\hat{y}}
\newcommand \hY{\hat{Y}}
\newcommand \e{\varepsilon}
\newcommand \he{\hat{\e}}
\newcommand \z{z}
\newcommand \hVar{\widehat{\Var}}
\newcommand \hCorr{\widehat{\Corr}}
\newcommand \hCov{\widehat{\Cov}}
\newcommand \cN{\mathcal{N}}
\newcommand \RR{\mathbb{R}}
\newcommand \NN{\mathbb{N}}
\newcommand{\cF}{\mathcal{F}}
\newcommand{\cH}{\mathcal{H}}
\newcommand{\dBin}{\mathrm{Bin}}


\begin{document}


\begin{enumerate}
    \item 
        В куче лежит 6666 камней.
        Бульбазавр и Пикачу берут камни из кучи по очереди.
        Бульбазавр берёт камень первым.
        Бульбазавр может взять 2, 3 или 5 камней за один ход.
        Пикачу может взять 1 или 4 камня за один ход.
        
        Проигрывает игру тот, кто первым не сможет сделать ход по правилам.
        \begin{enumerate}
         \item Сможет ли Бульбазавр выиграть?
         \item Если Бульбазавр может выиграть, то какой первый ход ему нужно сделать?
        \end{enumerate}
    \item У кота есть стартовый запас рыбы $y_1 \in [0;1]$.
    В момент времени $t$ кот может выловить и съесть количество рыбы $x_t \in [0; y_t]$.
    Мгновенное удовольствие кота равно $u_t = 3 + 2\ln x_t$.
    Оставшаяся невыловленная рыба размножается согласно правилу $y_{t+1} = (y_t - x_t)^{0.5}$.

    Кот выбирает величины $x_1$, $x_2$, \ldots, чтобы максимизировать 
    суммарную полезность с дисконт-фактором $0.5$:
    \[
     V_1 = \sum_{t=1}^{\infty} 0.5^{t-1} u_t \to \max.
    \]

    \begin{enumerate}
        \item Запишите уравнение Беллмана для этой задачи. 
        \item Найдите оптимальное потребление в каждом периоде. 
    \end{enumerate}
    
 
 
    Подсказка: можно предположить, что функция ценности имеет вид $V(y) = a + b \ln y$.


\end{enumerate}


\newpage

\begin{enumerate}
    \item 
        В куче лежит 5555 камней.
        Бульбазавр и Пикачу берут камни из кучи по очереди.
        Бульбазавр берёт камень первым.
        Бульбазавр может взять 2, 3 или 5 камней за один ход.
        Пикачу может взять 1 или 4 камня за один ход.
        
        Проигрывает игру тот, кто первым не сможет сделать ход по правилам.
        \begin{enumerate}
         \item Сможет ли Бульбазавр выиграть?
         \item Если Бульбазавр может выиграть, то какой первый ход ему нужно сделать?
        \end{enumerate}
    \item У кота есть стартовый запас рыбы $y_1 \in [0;1]$.
    В момент времени $t$ кот может выловить и съесть количество рыбы $x_t \in [0; y_t]$.
    Мгновенное удовольствие кота равно $u_t = 3 + 5\ln x_t$.
    Оставшаяся невыловленная рыба размножается согласно правилу $y_{t+1} = (y_t - x_t)^{0.5}$.

    Кот выбирает величины $x_1$, $x_2$, \ldots, чтобы максимизировать 
    суммарную полезность с дисконт-фактором $0.5$:
    \[
     V_1 = \sum_{t=1}^{\infty} 0.5^{t-1} u_t \to \max.
    \]

    \begin{enumerate}
        \item Запишите уравнение Беллмана для этой задачи. 
        \item Найдите оптимальное потребление в каждом периоде. 
    \end{enumerate}
    
 
 
    Подсказка: можно предположить, что функция ценности имеет вид $V(y) = a + b \ln y$.


\end{enumerate}


\newpage

\begin{enumerate}
    \item 
        В куче лежит 7777 камней.
        Бульбазавр и Пикачу берут камни из кучи по очереди.
        Бульбазавр берёт камень первым.
        Бульбазавр может взять 2, 3 или 5 камней за один ход.
        Пикачу может взять 1 или 4 камня за один ход.
        
        Проигрывает игру тот, кто первым не сможет сделать ход по правилам.
        \begin{enumerate}
         \item Сможет ли Бульбазавр выиграть?
         \item Если Бульбазавр может выиграть, то какой первый ход ему нужно сделать?
        \end{enumerate}
    \item У кота есть стартовый запас рыбы $y_1 \in [0;1]$.
    В момент времени $t$ кот может выловить и съесть количество рыбы $x_t \in [0; y_t]$.
    Мгновенное удовольствие кота равно $u_t = 2 + 3\ln x_t$.
    Оставшаяся невыловленная рыба размножается согласно правилу $y_{t+1} = (y_t - x_t)^{0.5}$.

    Кот выбирает величины $x_1$, $x_2$, \ldots, чтобы максимизировать 
    суммарную полезность с дисконт-фактором $0.5$:
    \[
     V_1 = \sum_{t=1}^{\infty} 0.5^{t-1} u_t \to \max.
    \]

    \begin{enumerate}
        \item Запишите уравнение Беллмана для этой задачи. 
        \item Найдите оптимальное потребление в каждом периоде. 
    \end{enumerate}
    
 
 
    Подсказка: можно предположить, что функция ценности имеет вид $V(y) = a + b \ln y$.


\end{enumerate}


\newpage

\begin{enumerate}
    \item 
        В куче лежит 6677 камней.
        Бульбазавр и Пикачу берут камни из кучи по очереди.
        Бульбазавр берёт камень первым.
        Бульбазавр может взять 2, 3 или 5 камней за один ход.
        Пикачу может взять 1 или 4 камня за один ход.
        
        Проигрывает игру тот, кто первым не сможет сделать ход по правилам.
        \begin{enumerate}
         \item Сможет ли Бульбазавр выиграть?
         \item Если Бульбазавр может выиграть, то какой первый ход ему нужно сделать?
        \end{enumerate}
    \item У кота есть стартовый запас рыбы $y_1 \in [0;1]$.
    В момент времени $t$ кот может выловить и съесть количество рыбы $x_t \in [0; y_t]$.
    Мгновенное удовольствие кота равно $u_t = 5 + 2\ln x_t$.
    Оставшаяся невыловленная рыба размножается согласно правилу $y_{t+1} = (y_t - x_t)^{0.5}$.

    Кот выбирает величины $x_1$, $x_2$, \ldots, чтобы максимизировать 
    суммарную полезность с дисконт-фактором $0.5$:
    \[
     V_1 = \sum_{t=1}^{\infty} 0.5^{t-1} u_t \to \max.
    \]

    \begin{enumerate}
        \item Запишите уравнение Беллмана для этой задачи. 
        \item Найдите оптимальное потребление в каждом периоде. 
    \end{enumerate}
    
 
 
    Подсказка: можно предположить, что функция ценности имеет вид $V(y) = a + b \ln y$.


\end{enumerate}


\newpage

\begin{enumerate}
    \item 
        В куче лежит 7766 камней.
        Бульбазавр и Пикачу берут камни из кучи по очереди.
        Бульбазавр берёт камень первым.
        Бульбазавр может взять 2, 3 или 5 камней за один ход.
        Пикачу может взять 1 или 4 камня за один ход.
        
        Проигрывает игру тот, кто первым не сможет сделать ход по правилам.
        \begin{enumerate}
         \item Сможет ли Бульбазавр выиграть?
         \item Если Бульбазавр может выиграть, то какой первый ход ему нужно сделать?
        \end{enumerate}
    \item У кота есть стартовый запас рыбы $y_1 \in [0;1]$.
    В момент времени $t$ кот может выловить и съесть количество рыбы $x_t \in [0; y_t]$.
    Мгновенное удовольствие кота равно $u_t = 5 + 3\ln x_t$.
    Оставшаяся невыловленная рыба размножается согласно правилу $y_{t+1} = (y_t - x_t)^{0.5}$.

    Кот выбирает величины $x_1$, $x_2$, \ldots, чтобы максимизировать 
    суммарную полезность с дисконт-фактором $0.5$:
    \[
     V_1 = \sum_{t=1}^{\infty} 0.5^{t-1} u_t \to \max.
    \]

    \begin{enumerate}
        \item Запишите уравнение Беллмана для этой задачи. 
        \item Найдите оптимальное потребление в каждом периоде. 
    \end{enumerate}
    
 
 
    Подсказка: можно предположить, что функция ценности имеет вид $V(y) = a + b \ln y$.


\end{enumerate}




\end{document}

