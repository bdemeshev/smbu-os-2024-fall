% arara: xelatex
\documentclass[12pt]{article}

\usepackage{physics}

\usepackage{tikzducks}

\usepackage{tikz} % картинки в tikz
\usepackage{microtype} % свешивание пунктуации

\usepackage{array} % для столбцов фиксированной ширины

\usepackage{indentfirst} % отступ в первом параграфе

\usepackage{sectsty} % для центрирования названий частей
\allsectionsfont{\centering}

\usepackage{amsmath, amsfonts, amssymb} % куча стандартных математических плюшек

\usepackage{comment}

\usepackage[top=2cm, left=1.2cm, right=1.2cm, bottom=2cm]{geometry} % размер текста на странице

\usepackage{lastpage} % чтобы узнать номер последней страницы

\usepackage{enumitem} % дополнительные плюшки для списков
%  например \begin{enumerate}[resume] позволяет продолжить нумерацию в новом списке
\usepackage{caption}

\usepackage{url} % to use \url{link to web}

\newcommand{\smallduck}{\begin{tikzpicture}[scale=0.3]
    \duck[
        cape=black,
        hat=black,
        mask=black
    ]
    \end{tikzpicture}}

\usepackage{fancyhdr} % весёлые колонтитулы
\pagestyle{fancy}
\lhead{Методы оптимальных решений}
\chead{}
\rhead{Ответы и подсказки к задачам курса}
\lfoot{}
\cfoot{}
\rfoot{}
\renewcommand{\headrulewidth}{0.4pt}

\renewcommand{\footrulewidth}{0.4pt}

\usepackage{tcolorbox} % рамочки!

\usepackage{todonotes} % для вставки в документ заметок о том, что осталось сделать
% \todo{Здесь надо коэффициенты исправить}
% \missingfigure{Здесь будет Последний день Помпеи}
% \listoftodos - печатает все поставленные \todo'шки


\usepackage{framed} % для рамок и черты слева от минитеории, \leftbar


% более красивые таблицы
\usepackage{booktabs}
% заповеди из докупентации:
% 1. Не используйте вертикальные линни
% 2. Не используйте двойные линии
% 3. Единицы измерения - в шапку таблицы
% 4. Не сокращайте .1 вместо 0.1
% 5. Повторяющееся значение повторяйте, а не говорите "то же"



\usepackage{fontspec}
\usepackage{polyglossia}

\setmainlanguage{russian}
\setotherlanguages{english}

% download "Linux Libertine" fonts:
% http://www.linuxlibertine.org/index.php?id=91&L=1
\setmainfont{Linux Libertine O} % or Helvetica, Arial, Cambria
% why do we need \newfontfamily:
% http://tex.stackexchange.com/questions/91507/
\newfontfamily{\cyrillicfonttt}{Linux Libertine O}

\AddEnumerateCounter{\asbuk}{\russian@alph}{щ} % для списков с русскими буквами
\setlist[enumerate, 2]{label=\asbuk*),ref=\asbuk*}

%% эконометрические сокращения
\DeclareMathOperator{\Cov}{\mathbb{C}ov}
\DeclareMathOperator{\Corr}{\mathbb{C}orr}
\DeclareMathOperator{\Var}{\mathbb{V}ar}

\let\P\relax
\DeclareMathOperator{\P}{\mathbb{P}}

\DeclareMathOperator{\E}{\mathbb{E}}
% \DeclareMathOperator{\tr}{trace}
\DeclareMathOperator{\card}{card}
\DeclareMathOperator{\plim}{plim}
\DeclareMathOperator{\pCorr}{\mathrm{p}\mathbb{C}\mathrm{orr}}


\newcommand \hb{\hat{\beta}}
\newcommand \hs{\hat{\sigma}}
\newcommand \htheta{\hat{\theta}}
\newcommand \s{\sigma}
\newcommand \hy{\hat{y}}
\newcommand \hY{\hat{Y}}
\newcommand \e{\varepsilon}
\newcommand \he{\hat{\e}}
\newcommand \z{z}
\newcommand \hVar{\widehat{\Var}}
\newcommand \hCorr{\widehat{\Corr}}
\newcommand \hCov{\widehat{\Cov}}
\newcommand \cN{\mathcal{N}}
\newcommand \RR{\mathbb{R}}
\newcommand \NN{\mathbb{N}}
\newcommand{\cF}{\mathcal{F}}
\newcommand{\cH}{\mathcal{H}}


\newcommand{\dBern}{\mathrm{Bern}}
\newcommand{\dPois}{\mathrm{Pois}}
\newcommand{\dBin}{\mathrm{Bin}}
\newcommand{\dMult}{\mathrm{Mult}}
\newcommand{\dGeom}{\mathrm{Geom}}
\newcommand{\dNHGeom}{\mathrm{NHGeom}}
\newcommand{\dHGeom}{\mathrm{HGeom}}
\newcommand{\dDUnif}{\mathrm{DUnif}}
\newcommand{\dFS}{\mathrm{FS}}
\newcommand{\dNBin}{\mathrm{NBin}}

\newcommand{\dTri}{\mathrm{Triangle}}
\newcommand{\dUnif}{\mathrm{Unif}}
\newcommand{\dU}{\mathrm{U}}
\newcommand{\dCauchy}{\mathrm{Cauchy}}
\newcommand{\dN}{\mathcal{N}}
\newcommand{\dLN}{\mathcal{LN}}
\newcommand{\dExpo}{\mathrm{Expo}} % o is probably great to avoid confusion with exp function
\newcommand{\dExp}{\dExpo}
\newcommand{\dBeta}{\mathrm{Beta}}
\newcommand{\dGamma}{\mathrm{Gamma}}
\newcommand{\dWei}{\mathrm{Wei}}
\newcommand{\dLogistic}{\mathrm{Logistic}}
\newcommand{\dRayleigh}{\mathrm{Rayleigh}}
\newcommand{\dPareto}{\mathrm{Pareto}}

\DeclareMathOperator{\Convex}{Convex}
\DeclareMathOperator{\Hull}{Hull}
\DeclareMathOperator{\hull}{\Hull}
\DeclareMathOperator{\Span}{Span}
\DeclareMathOperator{\cone}{Cone}


\begin{document}

В решениях используются обозначения
\begin{leftbar}
  \noindent
  Линейная оболочка (linear span):
  \[
    \Span(v_1, v_2, v_3) = \left\{\alpha_1 v_1 + \alpha_2 v_2 + \alpha_3 v_3 \mid \alpha_1 \in \RR, \alpha_2 \in \RR, \alpha_3 \in \RR \right\}
  \]
  Конус (cone):
  \[
    \cone(v_1, v_2, v_3) = \left\{\alpha_1 v_1 + \alpha_2 v_2 + \alpha_3 v_3 \mid \alpha_1 \geq 0, \alpha_2 \geq 0, \alpha_3 \geq 0 \right\}
  \]
  Выпуклая линейная оболочка (convex linear hull):
  \[
    \Hull(v_1, v_2, v_3) = \Convex(v_1, v_2, v_3) = \left\{\alpha_1 v_1 + \alpha_2 v_2 + \alpha_3 v_3 \mid \alpha_1 \geq 0, \alpha_2 \geq 0, \alpha_3 \geq 0, \sum \alpha_i = 1 \right\}
  \]
\end{leftbar}
  

\section{Графические методы}

\begin{enumerate}
  \item 
    \begin{enumerate}
      \item 
      
      \item 
      \begin{align*}
        3 a_1 - 3 b_1 + x_2 \to \max \\
        2 a_1 - 2 b_1 + x_2 - x_3 = 8 \\
        -2 a_1 + 2 b_1 + x_2 - x_4 = -4 \\
        a_1 - b_1 + x_2 + x_5 = 11 \\
        a_1 \geq 0, b_1 \geq 0, x_2 \geq 0, 
        x_3 \geq 0, x_4 \geq 0, x_5 \geq 0  \\
      \end{align*}
    \end{enumerate}
  \item 
  \begin{enumerate}
    \item 
    
    \item 
    \begin{align*}
      x_1 + x_2 \to \max \\
      -x_1 + x_2 + x_3 = 4 \\
      -x_1 - 2x_2 + x_4 = -14 \\
      x_1 + x_2 + x_5 = 10 \\
      x_1 - 2x_2 + x_6 = 10 \\
      x_1 \geq 0, x_2 \geq 0, 
      x_3 \geq 0, x_4 \geq 0, x_5 \geq 0, x_6 \geq 0  \\
    \end{align*}
  \end{enumerate}

  \item 
  \begin{enumerate}
    \item $P_1 = (4, 8)  \in \hull(A, B, C)$, $P_2 = (2, 7)  \in \hull(A, B, C)$, 
    $P_3 = (5, 7) \in \hull(A, B, C)$, $P_4 = (9, 3) \notin \hull(A, B, C)$,
    $P_5 = (8, 4) \in \hull(A, B, C)$, $P_6 = (5, 6)  \in \hull(A, B, C)$. 
    \item Допустимое множество $\hull(A, B, C)$ является треугольником. 
    Все точки из множества $\hull(A, B, C)$ могут быть представлены в виде выпуклой линейной комбинации единственным образом. 
    \item 
    \item 
  \end{enumerate}
  \item 
  \begin{enumerate}
    \item 
    \item Допустимое множество $\hull(A, B, C)$ является треугольником. 
    Все точки из множества $\hull(A, B, C)$ могут быть представлены в виде выпуклой линейной комбинации единственным образом. 
    \item 
    \item 
  \end{enumerate}

  \item 
  \begin{enumerate}
    \item $P_1 = (0, 1) = C \in \hull(A, B, C)$, $P_2 = (8, 9) = B \in \hull(A, B, C)$, 
    $P_3 = (5, 8) \notin \hull(A, B, C)$, $P_4 = (4, 7) \in \hull(A, B, C)$,
    $P_5 = (3, 5) \in \hull(A, B, C)$, $P_6 = (8, 9) = A  \in \hull(A, B, C)$. 
    \item Допустимое множество $\hull(A, B, C)$ является треугольником. 
    Все точки из множества $\hull(A, B, C)$ могут быть представлены в виде выпуклой линейной комбинации единственным образом. 
    \item При $c > -1/2$ оптимум находится в точке $B$. 
    При $c < -1/2$ оптимум находится в точке $A$. 
    При $c = -1/2$ оптимум находится на отрезке $[A, B]$.
    \item При $a \leq -6$ задача является неограниченной. 
    При $a > -6$ задача является ограниченной. 
  \end{enumerate}
\end{enumerate}

\end{document}

