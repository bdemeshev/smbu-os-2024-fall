% arara: xelatex
\documentclass[12pt]{article}

\usepackage{physics}

% \usepackage{mathrsfs}

\usepackage{tikzducks}
\usepackage{pgfplots}
\pgfplotsset{compat=1.18}



\usepackage{tikz} % картинки в tikz
\usetikzlibrary{arrows}

\usepackage{microtype} % свешивание пунктуации

\usepackage{array} % для столбцов фиксированной ширины

\usepackage{indentfirst} % отступ в первом параграфе

\usepackage{sectsty} % для центрирования названий частей
\allsectionsfont{\centering}

\usepackage{amsmath, amsfonts, amssymb} % куча стандартных математических плюшек

\usepackage{comment}

\usepackage[top=2cm, left=1.2cm, right=1.2cm, bottom=2cm]{geometry} % размер текста на странице

\usepackage{lastpage} % чтобы узнать номер последней страницы

\usepackage{enumitem} % дополнительные плюшки для списков
%  например \begin{enumerate}[resume] позволяет продолжить нумерацию в новом списке
\usepackage{caption}

\usepackage{url} % to use \url{link to web}

\newcommand{\smallduck}{\begin{tikzpicture}[scale=0.3]
    \duck[
        cape=black,
        hat=black,
        mask=black
    ]
    \end{tikzpicture}}

\usepackage{fancyhdr} % весёлые колонтитулы
\pagestyle{fancy}
\lhead{Методы оптимальных решений}
\chead{}
\rhead{Ответы и подсказки к задачам курса}
\lfoot{}
\cfoot{}
\rfoot{}
\renewcommand{\headrulewidth}{0.4pt}

\renewcommand{\footrulewidth}{0.4pt}

\usepackage{tcolorbox} % рамочки!

\usepackage{todonotes} % для вставки в документ заметок о том, что осталось сделать
% \todo{Здесь надо коэффициенты исправить}
% \missingfigure{Здесь будет Последний день Помпеи}
% \listoftodos - печатает все поставленные \todo'шки


\usepackage{framed} % для рамок и черты слева от минитеории, \leftbar


% более красивые таблицы
\usepackage{booktabs}
% заповеди из докупентации:
% 1. Не используйте вертикальные линни
% 2. Не используйте двойные линии
% 3. Единицы измерения - в шапку таблицы
% 4. Не сокращайте .1 вместо 0.1
% 5. Повторяющееся значение повторяйте, а не говорите "то же"



\usepackage{fontspec}
\usepackage{polyglossia}

\setmainlanguage{russian}
\setotherlanguages{english}

% download "Linux Libertine" fonts:
% http://www.linuxlibertine.org/index.php?id=91&L=1
\setmainfont{Linux Libertine O} % or Helvetica, Arial, Cambria
% why do we need \newfontfamily:
% http://tex.stackexchange.com/questions/91507/
\newfontfamily{\cyrillicfonttt}{Linux Libertine O}

\AddEnumerateCounter{\asbuk}{\russian@alph}{щ} % для списков с русскими буквами
\setlist[enumerate, 2]{label=\asbuk*),ref=\asbuk*}

%% эконометрические сокращения
\DeclareMathOperator{\Cov}{\mathbb{C}ov}
\DeclareMathOperator{\Corr}{\mathbb{C}orr}
\DeclareMathOperator{\Var}{\mathbb{V}ar}

\let\P\relax
\DeclareMathOperator{\P}{\mathbb{P}}

\DeclareMathOperator{\E}{\mathbb{E}}
% \DeclareMathOperator{\tr}{trace}
\DeclareMathOperator{\card}{card}
\DeclareMathOperator{\plim}{plim}
\DeclareMathOperator{\pCorr}{\mathrm{p}\mathbb{C}\mathrm{orr}}


\newcommand \hb{\hat{\beta}}
\newcommand \hs{\hat{\sigma}}
\newcommand \htheta{\hat{\theta}}
\newcommand \s{\sigma}
\newcommand \hy{\hat{y}}
\newcommand \hY{\hat{Y}}
\newcommand \e{\varepsilon}
\newcommand \he{\hat{\e}}
\newcommand \z{z}
\newcommand \hVar{\widehat{\Var}}
\newcommand \hCorr{\widehat{\Corr}}
\newcommand \hCov{\widehat{\Cov}}
\newcommand \cN{\mathcal{N}}
\newcommand \RR{\mathbb{R}}
\newcommand \NN{\mathbb{N}}
\newcommand{\cF}{\mathcal{F}}
\newcommand{\cH}{\mathcal{H}}


\newcommand{\dBern}{\mathrm{Bern}}
\newcommand{\dPois}{\mathrm{Pois}}
\newcommand{\dBin}{\mathrm{Bin}}
\newcommand{\dMult}{\mathrm{Mult}}
\newcommand{\dGeom}{\mathrm{Geom}}
\newcommand{\dNHGeom}{\mathrm{NHGeom}}
\newcommand{\dHGeom}{\mathrm{HGeom}}
\newcommand{\dDUnif}{\mathrm{DUnif}}
\newcommand{\dFS}{\mathrm{FS}}
\newcommand{\dNBin}{\mathrm{NBin}}

\newcommand{\dTri}{\mathrm{Triangle}}
\newcommand{\dUnif}{\mathrm{Unif}}
\newcommand{\dU}{\mathrm{U}}
\newcommand{\dCauchy}{\mathrm{Cauchy}}
\newcommand{\dN}{\mathcal{N}}
\newcommand{\dLN}{\mathcal{LN}}
\newcommand{\dExpo}{\mathrm{Expo}} % o is probably great to avoid confusion with exp function
\newcommand{\dExp}{\dExpo}
\newcommand{\dBeta}{\mathrm{Beta}}
\newcommand{\dGamma}{\mathrm{Gamma}}
\newcommand{\dWei}{\mathrm{Wei}}
\newcommand{\dLogistic}{\mathrm{Logistic}}
\newcommand{\dRayleigh}{\mathrm{Rayleigh}}
\newcommand{\dPareto}{\mathrm{Pareto}}

\DeclareMathOperator{\Convex}{Convex}
\DeclareMathOperator{\Hull}{Hull}
\DeclareMathOperator{\hull}{\Hull}
\DeclareMathOperator{\Span}{Span}
\DeclareMathOperator{\cone}{Cone}


\begin{document}

В решениях используются обозначения
\begin{leftbar}
  \noindent
  Линейная оболочка (linear span):
  \[
    \Span(v_1, v_2, v_3) = \left\{\alpha_1 v_1 + \alpha_2 v_2 + \alpha_3 v_3 \mid \alpha_1 \in \RR, \alpha_2 \in \RR, \alpha_3 \in \RR \right\}
  \]
  Конус (cone):
  \[
    \cone(v_1, v_2, v_3) = \left\{\alpha_1 v_1 + \alpha_2 v_2 + \alpha_3 v_3 \mid \alpha_1 \geq 0, \alpha_2 \geq 0, \alpha_3 \geq 0 \right\}
  \]
  Выпуклая линейная оболочка (convex linear hull):
  \[
    \Hull(v_1, v_2, v_3) = \Convex(v_1, v_2, v_3) = \left\{\alpha_1 v_1 + \alpha_2 v_2 + \alpha_3 v_3 \mid \alpha_1 \geq 0, \alpha_2 \geq 0, \alpha_3 \geq 0, \sum \alpha_i = 1 \right\}
  \]
\end{leftbar}
  

\section*{Графические методы}

\begin{enumerate}
  \item 
    \begin{enumerate}
      \item Оптимум: $(x_1 = 5, x_2=6)$, $z=21$.
      
      \item 
      \begin{align*}
        3 a_1 - 3 b_1 + x_2 \to \max \\
        2 a_1 - 2 b_1 + x_2 - x_3 = 8 \\
        -2 a_1 + 2 b_1 + x_2 - x_4 = -4 \\
        a_1 - b_1 + x_2 + x_5 = 11 \\
        a_1 \geq 0, b_1 \geq 0, x_2 \geq 0, 
        x_3 \geq 0, x_4 \geq 0, x_5 \geq 0  \\
      \end{align*}
    \end{enumerate}
  \item 
  \begin{enumerate}
    \item Оптимум: $(x_1 = 2, x_2=6)$, $z=2$.
    

    \definecolor{zzttqq}{rgb}{0.6,0.2,0}
    \definecolor{ududff}{rgb}{0.30196078431372547,0.30196078431372547,1}
    \definecolor{uuuuuu}{rgb}{0.26666666666666666,0.26666666666666666,0.26666666666666666}
    \definecolor{qqwuqq}{rgb}{0,0.39215686274509803,0}
    
    \begin{tikzpicture}[line cap=round,line join=round,>=triangle 45,x=1cm,y=1cm, scale=0.5]
    \begin{axis}[
    x=1cm,y=1cm,
    axis lines=middle,
    ymajorgrids=true,
    xmajorgrids=true,
    xmin=-1.620000000000001,
    xmax=19.12666666666668,
    ymin=-1.4400000000000006,
    ymax=11.879999999999992,
    xtick={-1,0,...,19},
    ytick={-1,0,...,11},]
    \clip(-1.62,-1.44) rectangle (19.12666666666668,11.88);
    \fill[line width=2pt,color=zzttqq,fill=zzttqq,fill opacity=0.10000000149011612] (3,7) -- (2,6) -- (6,4) -- cycle;
    \draw [line width=2pt,color=qqwuqq,domain=-1.62:19.12666666666668] plot(\x,{(--4--1*\x)/1});
    \draw [line width=2pt,domain=-1.62:19.12666666666668] plot(\x,{(-14--1*\x)/-2});
    \draw [line width=2pt,domain=-1.62:19.12666666666668] plot(\x,{(--10-1*\x)/1});
    \draw [line width=2pt,domain=-1.62:19.12666666666668] plot(\x,{(--10-1*\x)/-2});
    \draw [line width=2pt,color=zzttqq] (3,7)-- (2,6);
    \draw [line width=2pt,color=zzttqq] (2,6)-- (6,4);
    \draw [line width=2pt,color=zzttqq] (6,4)-- (3,7);
    \begin{scriptsize}
    \draw[color=qqwuqq] (-1.3533333333333342,2.78) node {$eq1$};
    \draw[color=black] (-1.3533333333333342,7.68666666666666) node {$eq2$};
    \draw[color=black] (-1.3533333333333342,11.433333333333325) node {$eq3$};
    \draw[color=black] (7.353333333333338,-1.18) node {$eq4$};
    \draw [fill=uuuuuu] (3,7) circle (2pt);
    \draw[color=uuuuuu] (3.1,7.26) node {$A$};
    \draw [fill=uuuuuu] (2,6) circle (2pt);
    \draw[color=uuuuuu] (2.1,6.26) node {$B$};
    \draw [fill=ududff] (6,4) circle (2.5pt);
    \draw[color=ududff] (6.1,4.286666666666663) node {$C$};
    \draw[color=zzttqq] (2.3933333333333344,6.806666666666661) node {$c$};
    \draw[color=zzttqq] (3.9533333333333354,4.966666666666662) node {$a$};
    \draw[color=zzttqq] (4.7,5.806666666666662) node {$b$};
    \end{scriptsize}
    \end{axis}
    \end{tikzpicture}
    


    \item 
    \begin{align*}
      x_1 + x_2 \to \max \\
      -x_1 + x_2 + x_3 = 4 \\
      -x_1 - 2x_2 + x_4 = -14 \\
      x_1 + x_2 + x_5 = 10 \\
      x_1 - 2x_2 + x_6 = 10 \\
      x_1 \geq 0, x_2 \geq 0, 
      x_3 \geq 0, x_4 \geq 0, x_5 \geq 0, x_6 \geq 0  \\
    \end{align*}
  \end{enumerate}

  \item % 1.3
  \begin{enumerate}
    \item $P_1 = (4, 8)  \in \hull(A, B, C)$, $P_2 = (2, 7)  \in \hull(A, B, C)$, 
    $P_3 = (5, 7) \in \hull(A, B, C)$, $P_4 = (9, 3) \notin \hull(A, B, C)$,
    $P_5 = (8, 4) \in \hull(A, B, C)$, $P_6 = (5, 6)  \in \hull(A, B, C)$. 
    \item Допустимое множество $\hull(A, B, C)$, $A = (2, 7)$, $B = (4, 8)$, $C = (8, 4)$  является треугольником. 
    Все точки из множества $\hull(A, B, C)$ могут быть представлены в виде выпуклой линейной комбинации единственным образом. 
    \item При $c > 3$ оптимум находится в точке $B$. 
    При $c < 3$ оптимум находится в точке $C$. 
    При $c = 3$ оптимум находится на отрезке $[B, C]$.
    \item При $a \leq -2$ задача является неограниченной. 
    При $a > -2$ задача является ограниченной. 
    \item 
  \end{enumerate}
  \item % 1.4
  \begin{enumerate}
    \item $P_1 = (4, 0) \notin \hull(A, B, C)$, $P_2 = (1, 8) \in \hull(A, B, C)$, 
    $P_3 = (2, 4) \in \hull(A, B, C)$, $P_4 = (3, 8) \in \hull(A, B, C)$,
    $P_5 = (-3, 13) \notin \hull(A, B, C)$, $P_6 = (4, 4)   \in \hull(A, B, C)$. 
    \item Допустимое множество $\hull(A, B, C)$, $A = (-3, 14)$, $B = (5, 6)$, $C = (3, 2)$ является треугольником. 
    Все точки из множества $\hull(A, B, C)$ могут быть представлены в виде выпуклой линейной комбинации единственным образом. 
    \item При $c > 2$ оптимум находится в точке $B$. 
    При $c < 2$ оптимум находится в точке $A$. 
    При $c = 2$ оптимум находится на отрезке $[A, B]$.
    \item При $a \leq -2$ задача является неограниченной. 
    При $a > -2$ задача является ограниченной. 
    \item 
  \end{enumerate}

  \item % 1.5
  \begin{enumerate}
    \item $P_1 = (0, 1) = C \in \hull(A, B, C)$, $P_2 = (8, 9) = B \in \hull(A, B, C)$, 
    $P_3 = (5, 8) \notin \hull(A, B, C)$, $P_4 = (4, 7) \in \hull(A, B, C)$,
    $P_5 = (3, 5) \in \hull(A, B, C)$, $P_6 = (0, 5) = A  \in \hull(A, B, C)$. 
    \item Допустимое множество $\hull(A, B, C)$, $A = (0, 5)$, $B = (8, 9)$, $C = (0, 1)$ является треугольником. 
    Все точки из множества $\hull(A, B, C)$ могут быть представлены в виде выпуклой линейной комбинации единственным образом. 
    \item При $c > -1/2$ оптимум находится в точке $B$. 
    При $c < -1/2$ оптимум находится в точке $A$. 
    При $c = -1/2$ оптимум находится на отрезке $[A, B]$.
    \item При $a \leq -6$ задача является неограниченной. 
    При $a > -6$ задача является ограниченной. 
    \item 
  \end{enumerate}

\item 
\item 

\item Обозначим список пересекаемых множеств буквой $\cF$, в этой задаче $\cF = \{D_1, D_2, D_3, D_4\}$.

Пересечение всех множеств равно 
\[
  S = \cap_{D \in \cF} D.
\] 

Рассмотрим произвольные точки $A$ и $B$ из множества $S$. 

По определению пересечения множеств, точки $A$ и $B$ лежат в любом из пересекаемых множеств $D \in \cF$. 
Любое множество $D \in \cF$ по условию задачи выпуклое, поэтому $[A, B] \subseteq D$.

Отрезок $[A, B]$ лежит в любом множестве $D \in \cF$, поэтому отрезок $[A, B]$ лежит в пересечении множеств $S$.

\item 

\item % 1.10
Допустимое множество: $\hull(A, B, C, D)$.

$A = (0, 1) \leftrightarrow x= (0, 1, 3, 11, 0)$, 
$B = (4/3, 7/3) \leftrightarrow x = (4/3, 7/3, 0, 13/3, 0)$, 
$C = (18/7, 12/7) \leftrightarrow x = (18/7, 12/7, 0, 0, 13/7)$, 
$D = (3, 0) \leftrightarrow x = (3, 0, 3, 0, 4)$.

\item % 1.11
Допустимое множество: $\hull(A, B, C)$.


$A = (1, 0) \leftrightarrow x = (1, 0, 0, 8)$, 
$B = (3/ 11, 32/11) \leftrightarrow x = (3/11, 32/11, 0, 0)$, 
$C = (9, 0) \leftrightarrow x = (9, 0, 32, 0)$.

\end{enumerate}

\section*{Симплекс метод}

\begin{enumerate}
  \item 
\end{enumerate}

\section*{Двойственность}

\begin{enumerate}
  \item 
\end{enumerate}

\section*{Транспортная задача}

\begin{enumerate}
  \item 
\end{enumerate}

\end{document}

