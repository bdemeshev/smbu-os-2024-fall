% arara: xelatex
\documentclass[12pt]{article}

% \usepackage{physics}


\usepackage{tikzducks}

\usepackage{tikz} % картинки в tikz
\usepackage{microtype} % свешивание пунктуации

\usepackage{array} % для столбцов фиксированной ширины

\usepackage{indentfirst} % отступ в первом параграфе

\usepackage{sectsty} % для центрирования названий частей
\allsectionsfont{\centering}

\usepackage{amsmath, amsfonts, amssymb} % куча стандартных математических плюшек

\usepackage{comment}

\usepackage[top=2cm, left=1.2cm, right=1.2cm, bottom=2cm]{geometry} % размер текста на странице

\usepackage{lastpage} % чтобы узнать номер последней страницы

\usepackage{enumitem} % дополнительные плюшки для списков
%  например \begin{enumerate}[resume] позволяет продолжить нумерацию в новом списке
\usepackage{caption}

\usepackage{url} % to use \url{link to web}


\newcommand{\smallduck}{\begin{tikzpicture}[scale=0.3]
    \duck[
        cape=black,
        hat=black,
        mask=black
    ]
    \end{tikzpicture}}

\usepackage{fancyhdr} % весёлые колонтитулы
\pagestyle{fancy}
\lhead{}
\chead{}
\rhead{Ответы к экзамену}
\lfoot{Методы оптимальных решений}
\cfoot{}
\rfoot{}

\renewcommand{\headrulewidth}{0.4pt}
\renewcommand{\footrulewidth}{0.4pt}

\usepackage{tcolorbox} % рамочки!

\usepackage{todonotes} % для вставки в документ заметок о том, что осталось сделать
% \todo{Здесь надо коэффициенты исправить}
% \missingfigure{Здесь будет Последний день Помпеи}
% \listoftodos - печатает все поставленные \todo'шки


% более красивые таблицы
\usepackage{booktabs}
% заповеди из докупентации:
% 1. Не используйте вертикальные линни
% 2. Не используйте двойные линии
% 3. Единицы измерения - в шапку таблицы
% 4. Не сокращайте .1 вместо 0.1
% 5. Повторяющееся значение повторяйте, а не говорите "то же"


\setcounter{MaxMatrixCols}{20}
% by crazy default pmatrix supports only 10 cols :)


\usepackage{fontspec}
\usepackage{libertine}
\usepackage{polyglossia}

\setmainlanguage{russian}
\setotherlanguages{english}

% download "Linux Libertine" fonts:
% http://www.linuxlibertine.org/index.php?id=91&L=1
% \setmainfont{Linux Libertine O} % or Helvetica, Arial, Cambria
% why do we need \newfontfamily:
% http://tex.stackexchange.com/questions/91507/
% \newfontfamily{\cyrillicfonttt}{Linux Libertine O}

\AddEnumerateCounter{\asbuk}{\russian@alph}{щ} % для списков с русскими буквами
\setlist[enumerate, 2]{label=\asbuk*),ref=\asbuk*}

%% эконометрические сокращения
\DeclareMathOperator{\Cov}{\mathbb{C}ov}
\DeclareMathOperator{\Corr}{\mathbb{C}orr}
\DeclareMathOperator{\Var}{\mathbb{V}ar}
\DeclareMathOperator{\col}{col}
\DeclareMathOperator{\row}{row}

\let\P\relax
\DeclareMathOperator{\P}{\mathbb{P}}

\DeclareMathOperator{\E}{\mathbb{E}}
% \DeclareMathOperator{\tr}{trace}
\DeclareMathOperator{\card}{card}

\DeclareMathOperator{\Convex}{Convex}

\newcommand \cN{\mathcal{N}}
\newcommand \RR{\mathbb{R}}
\newcommand \NN{\mathbb{N}}



% transportation table 
% https://tex.stackexchange.com/questions/83713/how-to-make-a-transportation-tableau
\newcolumntype{C}{@{}c@{}}
\newcommand{\bottombox}[1]{\makebox[2em][r]{#1}\hspace*{\tabcolsep}\hspace*{2em}}%
\newcommand{\innerbox}[2]{%
    \begin{tabular}[b]{c|c}
       \rule{2em}{0pt}\rule[-2ex]{0pt}{5ex} & \makebox[0.9em]{\footnotesize{#2}} \\\cline{2-2}
       \multicolumn{2}{r}{{#1}\hspace*{1.5\tabcolsep}\hspace*{2em}\rule[-2ex]{0pt}{5ex}}
    \end{tabular}}
\renewcommand{\arraystretch}{1.25}

% circled text in math mode
% https://tex.stackexchange.com/questions/67127/creating-a-circled-operator-which-expands-into-a-lozenge
\makeatletter
\newcommand*\comp[2][]{%
  \ensuremath{%
    \mathbin{%
      \mathpalette{\comp@aux{#1}}{#2}%
    }%
  }%
}
\newdimen\comp@unit
\newcommand*{\comp@aux}[3]{%
  #2%
  \mskip.5\thinmuskip\nonscript\mskip-.25\thinmuskip
  \begingroup
    \sbox0{%
      $%
        \m@th % \mathsurround=0pt
        #2% \displaystyle, \textstyle, ...
        \mkern9mu %
      $%
    }%
  \edef\x{\endgroup
    \comp@unit=\the\wd0 %
  }\x
  \tikz[baseline=(char.base)]{%
    \node[
      rectangle,
      draw,
      minimum height=2\comp@unit,
      minimum width=2\comp@unit,
      rounded corners=\comp@unit,
      inner sep=.33\comp@unit,
      line width=.05\comp@unit,
      #1%
    ] (char) {%
      $%
        \m@th % \mathsurround=0pt
        #2% \displaystyle, \textstyle, ...
        \rule{0pt}{\comp@unit}%
        #3%
      $%
    };%
  }%
  \mskip.5\thinmuskip\nonscript\mskip-.25\thinmuskip
}
\makeatother



\begin{document}

\section*{Вариант 1}


\begin{enumerate}
  \item 
  \begin{enumerate}
    \item Оптимум равен 26, $3A + C$ или $2B + 2A$.
    \item Оптимум равен 26, $3A + C$ или $2B + 2A$.
    \item 
    \[
    \begin{cases}
      5x_a + 8x_b + 11 x_c \to \max \\
      3x_a + 5x_b + 7x_c \leq 16 \\
      x_a, x_b, x_c \in \{0, 1, 2, 3, \ldots \}        
    \end{cases}
    \]
  \end{enumerate}
  \item 
  \begin{enumerate}
    \item Оптимум — $\Convex(A, B)$, $94$.
    
    \begin{tabular}{c|C|C|C|}
  $A$  & $b_1 = 8$         & $b_2 = 10$             & $b_3 = 11$         \\\hline
$a_1 = 4$ & \innerbox{}{1}   & \innerbox{2}{2}   & \innerbox{2}{1}   \\\hline
$a_2 = 3$ & \innerbox{}{4} & \innerbox{3}{3}   & \innerbox{}{8}       \\\hline
$a_3 = 13$ & \innerbox{8}{1} & \innerbox{5}{7}    & \innerbox{}{6}       \\\hline
$a_4 = 9$ & \innerbox{}{3}   & \innerbox{}{6} & \innerbox{9}{4}   \\\hline
\end{tabular}
    
\begin{tabular}{c|C|C|C|}
  $B$  & $b_1 = 8$         & $b_2 = 10$             & $b_3 = 11$         \\\hline
$a_1 = 4$ & \innerbox{}{1}   & \innerbox{4}{2}   & \innerbox{}{1}   \\\hline
$a_2 = 3$ & \innerbox{}{4} & \innerbox{3}{3}   & \innerbox{}{8}       \\\hline
$a_3 = 13$ & \innerbox{8}{1} & \innerbox{3}{7}    & \innerbox{2}{6}       \\\hline
$a_4 = 9$ & \innerbox{}{3}   & \innerbox{}{6} & \innerbox{9}{4}   \\\hline
\end{tabular}

\item Систему можно записать в матричном виде $A x = b$, где 
\[
A = \begin{pmatrix}
  1 & 1 & 1 & 0 & 0 & 0 & 0 & 0 & 0 & 0 & 0 & 0 \\
  0 & 0 & 0 & 1 & 1 & 1 & 0 & 0 & 0 & 0 & 0 & 0 \\
  0 & 0 & 0 & 0 & 0 & 0 & 1 & 1 & 1 & 0 & 0 & 0 \\
  0 & 0 & 0 & 0 & 0 & 0 & 0 & 0 & 0 & 1 & 1 & 1 \\
  1 & 0 & 0 & 1 & 0 & 0 & 1 & 0 & 0 & 1 & 0 & 0 \\
  0 & 1 & 0 & 0 & 1 & 0 & 0 & 1 & 0 & 0 & 1 & 0 \\
  0 & 0 & 1 & 0 & 0 & 1 & 0 & 0 & 1 & 0 & 0 & 1 \\
\end{pmatrix}
\]

\item Оптимум — $\Convex(C, D)$, $126$.
    
\begin{tabular}{c|C|C|C|}
$A$  & $b_1 = 8$         & $b_2 = 10$             & $b_3 = 11$         \\\hline
$a_1 = 4$ & \innerbox{}{1}   & \innerbox{}{2}   & \innerbox{4}{1}   \\\hline
$a_2 = 3$ & \innerbox{}{4} & \innerbox{3}{3}   & \innerbox{}{8}       \\\hline
$a_3 = 13$ & \innerbox{}{$\infty$} & \innerbox{7}{7}    & \innerbox{6}{6}       \\\hline
$a_4 = 9$ & \innerbox{8}{3}   & \innerbox{}{6} & \innerbox{1}{4}   \\\hline
\end{tabular}

\begin{tabular}{c|C|C|C|}
$B$  & $b_1 = 8$         & $b_2 = 10$             & $b_3 = 11$         \\\hline
$a_1 = 4$ & \innerbox{}{1}   & \innerbox{4}{2}   & \innerbox{}{1}   \\\hline
$a_2 = 3$ & \innerbox{}{4} & \innerbox{3}{3}   & \innerbox{}{8}       \\\hline
$a_3 = 13$ & \innerbox{}{$\infty$} & \innerbox{3}{7}    & \innerbox{10}{6}       \\\hline
$a_4 = 9$ & \innerbox{8}{3}   & \innerbox{}{6} & \innerbox{1}{4}   \\\hline
\end{tabular}


\end{enumerate}
  \item 
\begin{enumerate}
  \item 8, $A_1 \to A_2 \to A_4 \to A_7 \to A_8$ или $A_1 \to A_2 \to A_4 \to A_7 \to A_6 \to A_8$.
  \item 4, $A_2 \to A_4 \to A_7$
  \item Матрица смежности $M$ состоит только из 0 или 1:
  \[
    M = \begin{pmatrix}
      0 & 1 & 1 & 1 & 0 & 0 & 0 & 0 \\
      0 & 0 & 1 & 1 & 1 & 0 & 0 & 0 \\
      0 & 0 & 0 & 1 & 1 & 1 & 1 & 0 \\
      0 & 0 & 0 & 0 & 0 & 0 & 1 & 0 \\
      0 & 0 & 0 & 0 & 0 & 0 & 0 & 1 \\
      0 & 0 & 0 & 0 & 1 & 0 & 0 & 1 \\
      0 & 0 & 0 & 0 & 0 & 1 & 0 & 1 \\
      0 & 0 & 0 & 0 & 0 & 0 & 0 & 0 \\
    \end{pmatrix}
    \]   
\end{enumerate}
\item 
\begin{enumerate}
  \item Допустимое множество $\Convex(A, B, C)$, $A = (-2, 6)$, $B = (2, 4)$, $C = (1, 10)$, оптимум равен 48.
  \item Оптимум 48, $x = (0, 1, 0, 5)$.
  \item $b_1 \leq 9/2$ или $\Delta b_1 \leq 3/2$.
\end{enumerate}
\item 

\begin{tabular}{ccccccc|c}
  \toprule
    & $x_1$ & $x_2$ & $x_3$ & $x_4$ & $x_5$ & $x_6$ & $b$  \\ \midrule
$x_2$ & 7 & 1 & 0 & 5 & 0 & -3 & 24  \\
$x_3$ & 1/2 & 0 & 1 & 1/2 & 0 & -3/2 & 3/2  \\
$x_5$ & 0 & 0 & 0 & -4 & 1 & -4 & 3  \\ \midrule
$\min z$ & -1 & 0 & 0 & -3 & 0 & 0 & $-5 - z$  \\ \bottomrule
\end{tabular}

Все решения: $x = (0, 24 + 3x_6, 3/2 + 3/2x_6, 0, 3 + 4x_6, x_6)$, $x_6 \geq 0$. 

\item 
\[
\begin{cases}
  21x_1 + 6x_2 + 7x_3 + 7(a_4 - b_4) \to \max \\
  6x_1 + 2x_2 + 3x_3 + 2(a_4 - b_4) = 24 \\
  x_1 + x_2 + x_3 + 5(a_4 - b_4) - x_5 = 12 \\
  x_1, x_2, x_3, a_4, b_4, x_5 \geq 0    
\end{cases}
\]

\item 
\begin{enumerate}
  \item Да, столбцы независимы:
  \[
  \col_5 A = \begin{pmatrix}
    0 \\ 
    2 \\ 
    3 \\
  \end{pmatrix}.
  \]
  \item Нет, столбцы зависимы:
  \[
  \col_1 A = \begin{pmatrix}
    1 \\
    2 \\
    4 \\
  \end{pmatrix}, \quad 
  \col_3 A = \begin{pmatrix}
    -1 \\
    0 \\
    -1 \\
  \end{pmatrix}, \quad 
  \col_5 A = \begin{pmatrix}
    0 \\ 
    2 \\ 
    3 \\
  \end{pmatrix}, \quad 
  \col_1 A + \col_3 A = \col_5 A
  \]
  \item Нет, $x_5  = -1$, а должно быть $x_5 \geq 0$. 
  \item Да, столбцы независимы:
  \[
    \col_1 A = \begin{pmatrix}
      1 \\
      2 \\
      4 \\
    \end{pmatrix}, \quad 
    \col_3 A = \begin{pmatrix}
      -1 \\
      0 \\
      -1 \\
    \end{pmatrix}
  \]

\end{enumerate}



\end{enumerate}

\section*{Вариант 2}


\begin{enumerate}
  \item 
  \begin{enumerate}
    \item Оптимум равен 21, $2A + C$ или $A + 2B$.
    \item Оптимум равен 21, $2A + C$ или $A + 2B$.
    \item 
    \[
    \begin{cases}
      5x_a + 8x_b + 11 x_c \to \max \\
      3x_a + 5x_b + 7x_c \leq 13 \\
      x_a, x_b, x_c \in \{0, 1, 2, 3, \ldots \}        
    \end{cases}
    \]
  \end{enumerate}
  \item 
  \begin{enumerate}
    \item Оптимум — $\Convex(A, B)$, $118$.
    
    \begin{tabular}{c|C|C|C|}
  $A$  & $b_1 = 8$         & $b_2 = 10$             & $b_3 = 11$         \\\hline
$a_1 = 4$ & \innerbox{}{3}   & \innerbox{2}{3}   & \innerbox{2}{2}   \\\hline
$a_2 = 3$ & \innerbox{}{6} & \innerbox{3}{4}   & \innerbox{}{9}       \\\hline
$a_3 = 13$ & \innerbox{8}{2} & \innerbox{5}{7}    & \innerbox{}{6}       \\\hline
$a_4 = 9$ & \innerbox{}{5}   & \innerbox{}{7} & \innerbox{9}{5}   \\\hline
\end{tabular}
    
\begin{tabular}{c|C|C|C|}
  $B$  & $b_1 = 8$         & $b_2 = 10$             & $b_3 = 11$         \\\hline
$a_1 = 4$ & \innerbox{}{3}   & \innerbox{4}{3}   & \innerbox{}{2}   \\\hline
$a_2 = 3$ & \innerbox{}{6} & \innerbox{3}{4}   & \innerbox{}{9}       \\\hline
$a_3 = 13$ & \innerbox{8}{2} & \innerbox{3}{7}    & \innerbox{2}{6}       \\\hline
$a_4 = 9$ & \innerbox{}{5}   & \innerbox{}{7} & \innerbox{9}{5}   \\\hline
\end{tabular}

\item Систему можно записать в матричном виде $A x = b$, где 
\[
A = \begin{pmatrix}
  1 & 1 & 1 & 0 & 0 & 0 & 0 & 0 & 0 & 0 & 0 & 0 \\
  0 & 0 & 0 & 1 & 1 & 1 & 0 & 0 & 0 & 0 & 0 & 0 \\
  0 & 0 & 0 & 0 & 0 & 0 & 1 & 1 & 1 & 0 & 0 & 0 \\
  0 & 0 & 0 & 0 & 0 & 0 & 0 & 0 & 0 & 1 & 1 & 1 \\
  1 & 0 & 0 & 1 & 0 & 0 & 1 & 0 & 0 & 1 & 0 & 0 \\
  0 & 1 & 0 & 0 & 1 & 0 & 0 & 1 & 0 & 0 & 1 & 0 \\
  0 & 0 & 1 & 0 & 0 & 1 & 0 & 0 & 1 & 0 & 0 & 1 \\
\end{pmatrix}
\]

\item Оптимум — $\Convex(A, B)$, $118$, не изменяется.
    
\begin{tabular}{c|C|C|C|}
  $A$  & $b_1 = 8$         & $b_2 = 10$             & $b_3 = 11$         \\\hline
$a_1 = 4$ & \innerbox{}{3}   & \innerbox{2}{3}   & \innerbox{2}{2}   \\\hline
$a_2 = 3$ & \innerbox{}{6} & \innerbox{3}{4}   & \innerbox{}{$\infty$}       \\\hline
$a_3 = 13$ & \innerbox{8}{2} & \innerbox{5}{7}    & \innerbox{}{6}       \\\hline
$a_4 = 9$ & \innerbox{}{5}   & \innerbox{}{7} & \innerbox{9}{5}   \\\hline
\end{tabular}
    
\begin{tabular}{c|C|C|C|}
  $B$  & $b_1 = 8$         & $b_2 = 10$             & $b_3 = 11$         \\\hline
$a_1 = 4$ & \innerbox{}{3}   & \innerbox{4}{3}   & \innerbox{}{2}   \\\hline
$a_2 = 3$ & \innerbox{}{6} & \innerbox{3}{4}   & \innerbox{}{$\infty$}       \\\hline
$a_3 = 13$ & \innerbox{8}{2} & \innerbox{3}{7}    & \innerbox{2}{6}       \\\hline
$a_4 = 9$ & \innerbox{}{5}   & \innerbox{}{7} & \innerbox{9}{5}   \\\hline
\end{tabular}


\end{enumerate}
  \item 
\begin{enumerate}
  \item 8, $A_1 \to A_2 \to A_4 \to A_7 \to A_8$ или $A_1 \to A_2 \to A_4 \to A_7 \to A_6 \to A_8$.
  \item 4, $A_2 \to A_4 \to A_7$
  \item Матрица смежности $M$ состоит только из 0 или 1:
  \[
  M = \begin{pmatrix}
    0 & 1 & 1 & 1 & 0 & 0 & 0 & 0 \\
    0 & 0 & 1 & 1 & 1 & 0 & 0 & 0 \\
    0 & 0 & 0 & 1 & 1 & 1 & 1 & 0 \\
    0 & 0 & 0 & 0 & 0 & 0 & 1 & 0 \\
    0 & 0 & 0 & 0 & 0 & 0 & 0 & 1 \\
    0 & 0 & 0 & 0 & 1 & 0 & 0 & 1 \\
    0 & 0 & 0 & 0 & 0 & 1 & 0 & 1 \\
    0 & 0 & 0 & 0 & 0 & 0 & 0 & 0 \\
  \end{pmatrix}
  \]
\end{enumerate}
\item 
\begin{enumerate}
  \item Допустимое множество $\Convex(A, B, C)$, $A = (-2, 6)$, $B = (2, 4)$, $C = (1, 10)$, оптимум равен 48.
  \item Оптимум 48, $x = (5, 1, 0, 0)$.
  \item $b_1 \leq 9/2$ или $\Delta b_1 \leq 3/2$.
\end{enumerate}
\item 

\begin{tabular}{ccccccc|c}
  \toprule
    & $x_1$ & $x_2$ & $x_3$ & $x_4$ & $x_5$ & $x_6$ & $b$  \\ \midrule
$x_2$ & 1 & 7 & 0 & 5 & -3 & 0 & 24  \\
$x_3$ & 0 & 1/2 & 1 & 1/2 & -3/2 & 0 & 3/2  \\
$x_5$ & 0 & 0 & 0 & -4 & -4 & 1 & 3  \\ \midrule
$\min z$ & 0 & -1 & 0 & -3 & 0 & 0 & $-5 - z$  \\ \bottomrule
\end{tabular}

Все решения: $x = (24 + 3x_5, 0, 3/2 + 3/2x_5, 0, x_5, 3 + 4x_5)$, $x_5 \geq 0$. 

\item 
\[
\begin{cases}
  6(a_1 - b_1) + 21x_2 + 7x_3 + 7x_4 \to \max \\
  2(a_1 - b_1) + 6x_2 + 3x_3 +2x_4 = 24 \\
  a_1 - b_1 + x_2 + x_3 + 5x_4 - x_5 = 12 \\
  a_1, b_1, x_2, x_3, x_4, x_5 \geq 0    
\end{cases}
\]

\item 
\begin{enumerate}
  \item Да, столбцы независимы:
  \[
  \col_5 A = \begin{pmatrix}
    0 \\ 
    2 \\ 
    3 \\
  \end{pmatrix}.
  \]
  \item Нет, столбцы зависимы:
  \[
  \col_1 A = \begin{pmatrix}
    1 \\
    2 \\
    4 \\
  \end{pmatrix}, \quad 
  \col_3 A = \begin{pmatrix}
    -1 \\
    0 \\
    -1 \\
  \end{pmatrix}, \quad 
  \col_5 A = \begin{pmatrix}
    0 \\ 
    2 \\ 
    3 \\
  \end{pmatrix}, \quad 
  \col_1 A + \col_3 A = \col_5 A
  \]
  \item Нет, $x_5  = -1$, а должно быть $x_5 \geq 0$. 
  \item Да, столбцы независимы:
  \[
    \col_1 A = \begin{pmatrix}
      1 \\
      2 \\
      4 \\
    \end{pmatrix}, \quad 
    \col_3 A = \begin{pmatrix}
      -1 \\
      0 \\
      -1 \\
    \end{pmatrix}
  \]

\end{enumerate}



\end{enumerate}


\end{document}

